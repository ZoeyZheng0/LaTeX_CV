\documentclass[a4,10pt]{article}

%%%%%%% --------------------------------------------------------------------------------------
%%%%%%%  STARTING HERE, DO NOT TOUCH ANYTHING 
%%%%%%% --------------------------------------------------------------------------------------

\usepackage{latexsym}
\usepackage[empty]{fullpage}
\usepackage{titlesec}
 \usepackage{marvosym}
\usepackage[usenames,dvipsnames]{color}
\usepackage{verbatim}
\usepackage[hidelinks]{hyperref}
\usepackage{fancyhdr}
\usepackage{multicol}
\usepackage{hyperref}
\usepackage{csquotes}
\usepackage{tabularx}
\hypersetup{colorlinks=true,urlcolor=blue}
\usepackage[11pt]{moresize}
\usepackage{setspace}
\usepackage{fontspec}
\usepackage[inline]{enumitem}
\usepackage{array}
\newcolumntype{P}[1]{>{\centering\arraybackslash}p{#1}}
\usepackage{anyfontsize}

%%%% Set Margins
\usepackage[margin=1cm, top=1cm]{geometry}

%%%% Set Fonts
\setmainfont[
BoldFont=SourceSansPro-Semibold.otf, %SourceSansPro-Bold.otf
ItalicFont=SourceSansPro-RegularIt.otf
]{SourceSansPro-Regular.otf}
\setsansfont{SourceSansPro-Semibold.otf}

%%%% Set Page
\pagestyle{fancy}
\fancyhf{} 
\fancyfoot{}
\renewcommand{\headrulewidth}{0pt}
\renewcommand{\footrulewidth}{0pt}

%%%% Set URL Style
\urlstyle{same}

%%%% Set Indentation
\raggedbottom
\raggedright
\setlength{\tabcolsep}{0in}

%%%% Set Secondary Color
\definecolor{UI_blue}{RGB}{32, 64, 151}

%%%% Define New Commands
\usepackage[style=nature, maxbibnames=6, sorting=none]{biblatex}
\addbibresource{Publications.bib}
\renewcommand*{\revsdnamepunct}{}

%%%% Bold Name in Publications
\renewcommand*{\mkbibnamegiven}[1]{%
\ifitemannotation{highlight}
{\textbf{#1}}
{#1}}

\renewcommand*{\mkbibnamefamily}[1]{%
\ifitemannotation{highlight}
{\textbf{#1}}
{#1}}

%%%% Set Sections formatting
\titleformat{\section}{
\color{UI_blue} \scshape \raggedright \large 
}{}{0em}{}[\vspace{-10pt} \hrulefill \vspace{-6pt}]

%%%% Set Subtext Formatting
\newcommand{\subtext}[1]{
#1\par\vspace{-0.2cm}}

% \newcommand{\subtextit}[1]{\vspace{0.15cm}
% \textit{ #1 \vspace{-0.2cm}} }

%%%% Set Item Spacing
\setlist[itemize]{align=parleft,left=0pt..1em}

%%%% New Itemize "Zitemize" Formatting - tighter spacing than itemize
\newenvironment{zitemize}{
\begin{itemize}\itemsep0pt \parskip0pt \parsep1pt}
{\end{itemize}\vspace{-0.5cm}}


%%%% Define Skills Bold Formatting
\newcommand{\hskills}[1]{
\textbf{\bfseries #1} }

%%%% Set Subsection formatting
\titleformat{\subsection}{\vspace{-0.1cm} 
\bfseries \fontsize{11pt}{2cm}}{}{0em}{}[\vspace{-0.2cm}]

%%%%%%% --------------------------------------------------------------------------------------
%%%%%%% --------------------------------------------------------------------------------------
%%%%%%%  END OF "DO NOT TOUCH" REGION
%%%%%%% --------------------------------------------------------------------------------------
%%%%%%% --------------------------------------------------------------------------------------

\begin{document}

%%%%%%% --------------------------------------------------------------------------------------
%%%%%%%  HEADER
%%%%%%% --------------------------------------------------------------------------------------
\begin{center}
    \begin{minipage}[b]{0.24\textwidth}
            % \large City, ST \\
            {\href{mailto:wanrongz@usc.edu}{wanrongz@usc.edu} } \\
            \href{https://ZoeyZheng0.github.io/}{https://ZoeyZheng0.github.io/}
    \end{minipage}% 
    \begin{minipage}[b]{0.52\textwidth}
            \centering
            {\LARGE Wanrong (Zoey) Zheng} \\ %
            % \vspace{0.05cm} \\
            % {\color{UI_blue} \Large{Person Re-Identification}} \\
    \end{minipage}% 
    \begin{minipage}[b]{0.24\textwidth}
            \flushright \large  %Willing to Relocate
            \normalsize (213) 706-3365 \\
            %\href{https://Add_your_portfolio_here_}{Portfolio}
            \normalsize Orchard Ave, Los Angeles
    \end{minipage}   


% {\color{UI_blue} \hrulefill}
\end{center}
% \vspace{-0.2cm}
%  Highly Driven Deep Learning Engineer with 3+ years of full-time experience handling the real world problems. \hskills{Areas of expertise:} Deep Learning, Data Analysis, Python. % \hskills{Impact:} \vspace{-0.2cm} 
%  Supplied facial verification algorithm for over 4
% hundred million smartphones. Win ICCV2021 Masked Face Recognition Challenge.
% \vspace{-0.2cm}


%%%%%%%  EDUCATION
%%%%%%% --------------------------------------------------------------------------------------
\section{Education }
\subsection*{University of Southern California, {\normalsize \normalfont Master of Science in Computer Science  \hfill \textit{Aug. 2020 --- May. 2023}}}
    \begin{zitemize}
        \item Advisor: Prof. \href{https://sites.usc.edu/iris-cvlab/professor-ram-nevatia/} {Ram Nevatia} \& Prof. \href{http://ilab.usc.edu/itti/} {Laurent Itti}
    \end{zitemize}
% \vspace{-0.15cm}
\subsection*{South University of Science and Technology of China, {\normalsize \normalfont Exchange in Computer Science \hfill \textit{Jan. 2021 --- Jun. 2021}}}
\vspace{0.15cm}
\subsection*{Anhui University, {\normalsize \normalfont Bachelor of Science in Computer Science, GPA: 89.07/100, top 6\%  \hfill \textit{Sep. 2014 --- Jun. 2018}}}
    % \begin{zitemize}
        % \item Rank: 22/357 (top \textbf{6\%}); All major courses of first three years achieve over 80 points out of 100
        % \item All major courses of first three years achieve over 80 points out of 100
        % \item Courses: Advanced Mathematics (100/100), Experiments of Computer Graphics (98/100)
    % \end{zitemize}
\vspace{0.2cm}

%%%%%%% --------------------------------------------------------------------------------------
%%%%%%%  RESEARCH INTERESTS
%%%%%%% --------------------------------------------------------------------------------------
\section{RESEARCH INTERESTS} 
\begin{zitemize}
    \item \label{*} Person Re-Identification [1, 2, 4]: retrieving the same individual based on the human pose, gait, and face information.
    \item \label{*} 3D vision [2]: representing and reconstructing 3-D dynamic human shape and pose.
    \item \label{*} Explainable Artificial Intelligence [3]: transparent and effective human-in-the-loop learning.
    \item \label{*} Face Recognition: high performance large scale Face Recognition task with real-world multi-race data.
\end{zitemize}

%%%%%%% --------------------------------------------------------------------------------------
%%%%%%%  PUBLICATIONS
%%%%%%% --------------------------------------------------------------------------------------
\section{Publications} 
\renewcommand\refname{\vskip -1.5em}
\nocite{*}
\printbibliography[heading=none]
\vspace{-0.5cm}


%%%%%%%  AWARDS & HONORS
%%%%%%% --------------------------------------------------------------------------------------
\section{Awards \& Honors}
\noindent \textbf{1st} on \href {https://arxiv.org/abs/1607.08221} {MS1M} dataset in \href {https://arxiv.org/abs/2108.08191} {Masked Face Recognition Challenge} (\textbf{ICCV 2021}) out of 136 teams \hfill{\textit{Oct. 2021}}

\noindent \textbf{2nd} on \href {https://paperswithcode.com/dataset/glint360k} {Glint360k} dataset in \href {https://arxiv.org/abs/2108.08191} {Masked Face Recognition Challenge} (\textbf{ICCV 2021}) out of 86 teams \hfill{\textit{Oct. 2021}}

\noindent National Endeavor Scholarship for Top Undergraduate Students of China (top 1\%) \hfill{\textit{Nov. 2017}}

\vspace{-0.2cm}
    
%%%%%%% --------------------------------------------------------------------------------------
%%%%%%%  ACADEMIC PROJECTS
%%%%%%% --------------------------------------------------------------------------------------
\section{Research Experience} %% (Or "Research", select as appropriate)

\subsection*{IRIS Computer Vision Lab, University of Southern California \hfill Los Angeles, CA}
\subtext{\textit{Research Assistant, Advisor: Prof. \href{https://sites.usc.edu/iris-cvlab/professor-ram-nevatia/} {Ram Nevatia} }\hfill \textit{Jan. 2022 --- Present}}
    \begin{zitemize}
        \item \textbf{GaitRef: Gait Recognition with Refined Sequential Skeletons
Knowledge Exchange}
            \begin{itemize}
                \item Combined the silhouettes and skeletons information and refined the framewise joint predictions for gait recognition.
                \item Utilized temporal information from silhouette sequences for refining the skeletons, without extra annotations, the refined skeletons achieved state-of-the-art gait recognition performance without extra annotations.
                \item On Gait3D, the proposed method outperformed the baseline by 6.1\% on Rank-1 and 5.4\% on Rank-5.
                \item Submitted one primary-author paper to CVPR 2023 \cite{Pap1}.
            \end{itemize}

        \item \textbf{CAT-NeRF: Constancy-Aware Tx$^2$Former for Dynamic Body Modeling}
            \begin{itemize}
                \item Proposed a novel structure to combine two Transformer layers for reconstructing dynamic body shapes, which separated appearance constancy and uniqueness of videos. 
                \item Achieved a 30.3\% PSNR relative improvement on H36M, compared with the SOTA baseline method.
                \item Submitted one second-author paper to CVPR 2023 \cite{Pap2}.
            \end{itemize}
        
        \end{zitemize}

\subsection*{iLab, University of Southern California \hfill Los Angeles, CA}
\subtext{\textit{Research Assistant, Advisor: Prof. \href{http://ilab.usc.edu/itti/} {Laurent Itti} }\hfill \textit{Jan. 2022 --- Present}}
    \begin{zitemize}
        \item \textbf{Towards Generic Interface to Human-Neural Network
Knowledge Exchange}
            \begin{itemize}
                \item Proposed a pipeline for humans to directly interact with Neural Networks on a structural representation of visual concepts.
                \item Constructed Structural Concept Graphs (SCG), which is a reasoning logic mechanism of Neural Networks in classification tasks by utilizing reasonable concepts extractor and Graph reasoning Network.
                \item Humans could make decisions on the SCG and use SCG to guide the original Neural Network backward by knowledge distillation. 
                \item Accuracy increased by about 4\% improvement on target ImageNet classes without a drop on the other classes. 
                \item Submitted one second-author paper to Nature Machine Intelligence \cite{Pap3}.
            \end{itemize}
        
        % \item Feature Level Augmentation on Few-shot Learning
        %     \begin{itemize} 
        %         \item Extract fifteen different parts of each image dataset CUB and shuffle within same class to compose a new dataset. Build fifteen Resnet backbones to train different parts as base model, and add augmentation directly on feature level
        %         \item After finetuning on the refer dataset, the accuracy increases 8\% comparing to baseline
        %     \end{itemize}
        
        \end{zitemize}

%%% ----- Best way to write items (Credit - FAANGPath)
        % \item Achieved X\% growth for XYZ using A, B, and C skills.
        % \item Led XYZ which led to X\% of improvement in ABC
        % \item Developed XYZ that did A, B, and C using X, Y, and Z. 

%%%%%%% ----------------------------------- Role 1 ----------------------------------- %%%%%%%
\subsection*{Identity Verification, SenseTime \hfill Shenzhen, China} 
\subtext{\textit{Research Engineer, Advisor: Dr. \href{https://scholar.google.com/citations?user=20Its9kAAAAJ&hl=en} {Yichao Wu} \& Mr. \href{https://scholar.google.com/citations?user=Dqjnn0gAAAAJ&hl=zh-CN}{Ding Liang}}\hfill \textit{Jan. 2021 --- Aug. 2021}}
    \begin{zitemize}
        \item \textbf{Phone Unlock Facial Verification}
            \begin{itemize} 
                \item Built a multi-race and multi-factor (hat, glasses, makeup) testset as the evaluation testset to promote granularity of evaluation result.
                \item Implemented different image preprocessing approaches and found out the best crop and align way for phone recognition scene.
                \item Applied feature ensemble, Adaptation training, and hard data mining to enhance performance on a weak domain while keeping accuracy on others. 
                \item Achieved 1e-6FAR@recall 87.65\% (increased by 7.41\%) on African race subset.

            \end{itemize}
        \item \textbf{Knowledge Distillation Optimization}
            \begin{itemize} 
                \item Proposed a novel loss to evaluate knowledge distillation, which used the student network to reconstruct the teacher's hidden layer. 
                \item Calculated the Normalized L2 Loss between the teacher hidden layer and student hidden layer as knowledge distillation loss. 
                \item The 1e-5FAR@recall increased by 2.93\% on Chinese Face Unlock.
                % \item To ease up the influence of different preprocessing strategies' influence to training, get the mean of different crop size and location student feature to compare with teacher feature, smoothed and fastened the loss convergence procedure.
            \end{itemize}
    \end{zitemize}

\subsection*{Smart City Group, SenseTime \hfill Shenzhen, China} 
\subtext{\textit{Research Engineer, Advisor: Dr. \href{https://scholar.google.com/citations?user=aDf9fpkAAAAJ&hl=en} {Xiaoke Jiang}  \& Dr. \href{https://scholar.google.com/citations?user=rEYarG0AAAAJ&hl=en}{Junjie Yan} \hfill Dec. 2019 --- Dec. 2020}}
    \begin{zitemize}
        \item \textbf{Self-Separated Network to Align Parts for 3D Convolution in Video Person Re-Identification}
            \begin{itemize} 
                \item Trained the Self-Separated Network in supervised / semi-supervised / unsupervised ways, which proved the efficiency of the semi-supervised alignment strategies, which used the labels with the selected position. 
                \item Designed and visualized on both synthetic and real data to show that selected labels helped the attention classifiers to pay attention to the desired parts and had the ability to adjust mistaken pose estimation.
                \item Received a 15.5\% Rank-1 improvement on iLIDS compared to the fully supervised way. 
                \item Published one paper on AAAI 2021 \cite{Pap4}.
            \end{itemize}
            
        \item \textbf{A Spatial-Temporal Model to Aid Subway Face Verification With Mask}
            \begin{itemize}
                \item Collected a dataset from a running face verification system for subway stops, which showed 91\% of error records were with masks. 
                \item Leveraged the spatial-temporal pattern of humans to aid the masked face verification at the subway entrance.
                \item Modeled the behavior of passengers from their history of riding data and computed a joint verification score by combining the spatial-temporal score and visual score.
                \item The presented spatial-temporal pattern could be used to aid the verification which avoided 15.9\% of real-world hard cases.
            \end{itemize}

    \end{zitemize}


%%%%%%% --------------------------------------------------------------------------------------
%%%%%%%  WORK EXPERIENCE
%%%%%%% --------------------------------------------------------------------------------------
\section{Work}

\subsection*{SenseTime Research \hfill Shenzhen, China} 
\subtext{\textit{Algorithm Development Engineer, Advisor: Dr. \href{https://scholar.google.com/citations?user=20Its9kAAAAJ&hl=en} {Yichao Wu} \& Dr. \href{https://scholar.google.com/citations?user=aDf9fpkAAAAJ&hl=en} {Xiaoke Jiang}    \hfill Sep. 2019 --- Aug. 2021}}
    \begin{zitemize}
        \item Responsible for supplying face unlock models for major Chinese mobile phone manufacturers.
        \item Prepared three different size levels of models for various products' performance needs and used different training strategies.
        \item Big model achieved \textbf{1e-6FAR@recall 90\%} for different races, including Caucasian, African, Asian, Indian,  and Latino.
    \end{zitemize}

\subsection*{Chinese Academy of Science, Shenzhen Institutes of Advanced Technology \hfill Shenzhen, China} 
\subtext{\textit{Research Assistant
    \hfill Jul. 2018 --- Jun. 2019}}
    \begin{zitemize}
        \item Led a team of four to develop a multi-stage abnormal condition detection system for real-time baby monitoring, which could detect whether babies were sleeping, vomiting, or their faces were covered.
    \end{zitemize}

\subsection*{The Chinese University of Hong Kong, Shenzhen Research Institute \hfill Shenzhen, China} 
\subtext{\textit{Research Intern \hfill Dec. 2017 --- Jun. 2018} }
    \begin{zitemize}
        \item Designed and implement a visual tracking system for pedestrian detection and tracking. 
        \item Split the target bounding box into 64 patches which were desecrated by RGB and Gradient features.  
        \item Determined foreground and background description using random walk with restart simulations. 
        \item Incorporated spatially ordered and weighted patch descriptor into the structured output tracking framework.
    \end{zitemize}




% \vspace{-0.4cm}
\section{Patents}
\begin{enumerate}[leftmargin=*]
  \item \textbf{Wanrong Zheng}, Xiaoke Jiang, Jikui Bao, Qichen Li and Cong Ji. \textbf{A Railway Face Recognition Solution Based on History Passengers' Riding Pattern.}\textit{CN202110654499.8} (2021)
  
  \item \textbf{Wanrong Zheng} and Xiaoke Jiang. \textbf{A Identification method Based on History Passenger Flow Big Data.}\textit{CN202011132611.3} (2020)
  
  \item Sun Zhe and \textbf{Wanrong Zheng}. \textbf{Passenger Illegal Handing Bags Across Railing Detection in Real Railway Scene.} (2020)
  
\end{enumerate}
\end{document}